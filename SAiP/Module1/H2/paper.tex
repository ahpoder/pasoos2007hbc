\documentclass[a4paper,10pt]{article}

\usepackage[latin1]{inputenc}
\usepackage{epsfig}
\usepackage{graphicx}
\usepackage{url}
\usepackage{times}
\usepackage{rotating}
\usepackage{multirow}

\renewcommand{\descriptionlabel}[1]
    {\hspace{\labelsep}\emph{#1}}

\begin{document}

\title{Exercise 2: Quality Attributes and Architectural Design of the HS07 System}

\author{
  Anders H Poder, Jesper Dalberg, Lars Kringelbach\\\\
  Department of Computer Science, University of Aarhus\\
  Aabogade 34, 8200 {\AA}rhus N, Denmark\\\\
  \makeatletter
  \texttt{Group 11 - Kilo}\\
  \texttt{19951439, 20074976, 20074842}\\
  \texttt{\{ahp, jdalberg, u074842\}@daimi.au.dk}
}

\date{2008-02-21}

\maketitle

% =====================================================================
\begin{abstract}
  The HS07 system implements a closed-loop control of the heating in a
  private home. It monitors thermometers in the home, and based on
  measurements HS07 adjusts radiators in the home. This report gives a
  quality attribute analysis and discussion and a architectural design of 
  the HS07 system. The techniques used for both quality attributes and 
  architectural design are taken from \cite{bass2003sa}.
\end{abstract}

% =====================================================================
\section{Introduction}

Figure~\ref{fig:hs07} shows a schematic overview of HS07 in a
home. The home may be accessed by the home owner from the outside
through the HS07 gateway. The HS07 gateway also monitors and controls
the home.
\begin{figure}[!htb]
\centerline{\epsfig{figure=figures/hs07,scale=0.4 }}
\caption{HS07 in a home}
\label{fig:hs07}
\end{figure}

HS07 includes sensor and actuator hardware which runs on an embedded Java virtual
machine with standard software.

% =====================================================================
\section{Architectural Requirements}
\label{sec:requirements}

For our purposes there is one main use case for the HS07 system:
\begin{quote}
  \emph{Control Temperature}: The gateway collects measurements from
  thermometers and reports this to radiators that then control the
  temperature.
\end{quote}

The major driving quality attributes of the HS07 system
are:

\begin{itemize}
\item \emph{Performance.} HS07 should be performant so that a large
  number of thermometers and radiators may be part of the system.
\item \emph{Modifiability.} It must be possible to modify HS07 to
  include new types of sensors and actuators.
\item \emph{Usability.} It must be easy to use the gateway user interface.
  include new types of sensors and actuators.
\item \emph{Availability.} The system must be responsive and available for 
both users and hardware elements.
\item \emph{Safety.} The system must take care not to overheat the actuators.
\end{itemize}


% =====================================================================
\section{Quality Attributes}
The following chapters documents the selection of the Quality Attributes that we 
select for the HS07 system. The selection of QA is split into a 3-part process.
\begin{itemize}
\item Brainstorming - we think up scenarios
\item Selection - we chose which scenarios are interesting.
\item Refinement - we refine the selected scenarios
\end{itemize}
\subsection{Brainstorming}
In this part of the process we just attempt to think of as many possible scenarios for the HS07 system as we possibly can. 

\subsection{Scenarios}

\begin{enumerate}
\item A sensor stops responding.
\item Someone leaves a window open in an ice storm, and the sensors makes the actuators overheat.
\item Someone accidentally puts their stiletto through the Ethernet switch, disabling networking 
parts of the system.
\item After an average user has used the system for less than one hour e or she has mastered the basic functionality
\item The user adds a new actuator and the system configures to use the new hardware automatically within 1 minute.
\item The user logs on externally through the gateway and has access to the home control within 1 second of verifying 95\% of the time (5 seconds 4\% of the time, more than 5 seconds (unscheduled down-time) less than 1\%) is connected and has access
\item The user attempts to turn radiator above normal safety-level from outside the house and receive an error back explaining the infraction. The system stores a log of the episode and refuse to execute the command.
\item An attacker attempts to log on using an invalid password and is thrown off. The next attempt will take longer (progressive log-on time). The system stores a log of the failed attempt.
\item Developer updates and tests the code base with a new device type in 1 day.
\item A technical installer installs a system at a customer in no more than 4 hours.
\item Developer updates the code base to a new requirement in less than 30 man-hours
\item Configuration manager installs an upgrade remotely at a customer in no more than 30min
\item Tester/debugger extract debugging information directly from running system at customer.
\item The system handles 50 sensors and 50 actuators.
\item The system handles up to 50 rooms/groups with one or more sensors and actuators in each group.
\item The user changes the preferred temperature from a web page and the system adapts to it.
\item The customer requests that the system runs on other hardware (e.g. less power consuming hardware).
\item The customer requests that the system interfaces with other sensors/actuators?
\item The developer modifies the sensor poll interval at build time.
\item The system turns off all actuators when sensors cannot be probed.
\item The end user adds a sensor/actuator at runtime which registers automatically.
\end{enumerate}

\subsection{Selection}
Here we put the scenarios to the vote. This is done by emulation of a real-life situation where the actual stakeholder would
be part of the process. We take on the roles of the stakeholders in their absence, and chose which scenarios would be more
interesting for us.
\subsection{Stakeholders and interests}
\begin{itemize}
\item User (homeowner)
\begin{itemize}
\item Easy to use and configure
\item Always available
\item Safe (will not make an error and burn down the house)
\item Secure (other people cannot control my home)
\end{itemize}
\item Producer
\begin{itemize}
\item Cheap
\item Production line (modifiability for other uses)
\end{itemize}
\item Technical installer
\begin{itemize}
\item Easily deployable (installable)
\end{itemize}
\item Developer (?)
\begin{itemize}
\item Easily adjustable to new requirements
\item Scalable
\item Remote update (upgrading online)
\item Remote maintenance (debugging online)
\item Automatically testable (regression test)
\item Developing in parallel
\end{itemize}
\item Insurance company
\begin{itemize}
\item Safe with respect to fires (radiators) and break ins (door), ...
\end{itemize}
\end{itemize}

\subsection{The Vote}
\subsection{First vote}
\begin{tabular}{|l|c|c|c|c|c|c|c|}
\hline
Person & Stakeholder Role & Vote 1 & Vote 2 & Vote 3 & Vote 4 & Vote 5 & Vote 6\\
\hline
Jesper&User&4&5&6/16&20&21&2/7\\
Lars& & & & & & & \\
Anders& &2/7&4&6/16&8&14&5/21\\
\hline
\end{tabular}
\subsection{Second vote}
\begin{tabular}{|l|c|c|c|c|c|c|c|}
\hline
Person & Stakeholder Role & Vote 1 & Vote 2 & Vote 3 & Vote 4 & Vote 5 & Vote 6\\
\hline
Jesper& & & & & & & \\
Lars& & & & & & & \\
Anders& & & & & & & \\
\hline
\end{tabular}

\subsection{Refinement}
The process of refining the chosen scenarios centers around specification of various elements of the scenario. Stimulus, Response, Response Measurement. 
\begin{table}[!htp]
\begin{center}
\begin{tabular}{|p{0.3cm}|p{2.5cm}|p{8cm}|}
  \hline
  \multicolumn{2}{|p{3cm}|}{\bfseries Scenario(s):} & After an average user has used the system for less than one hour e or she has mastered the basic functionality \\
  \hline
  \multicolumn{2}{|p{3cm}|}{\bfseries Relevant Quality Attributes:} usability & \\
  \hline
  \multirow{6}{*}{\begin{sideways}{\bfseries Scenario Parts}\end{sideways}}
  & {\bfseries Source:} &  \\
  \cline{2-3}
  & {\bfseries Stimulus:} &  \\
  \cline{2-3}
  & {\bfseries Artifact} &  \\
  \cline{2-3}
  & {\bfseries Environment:} &  \\
  \cline{2-3}
  & {\bfseries Response:} &  \\
  \cline{2-3}
  & {\bfseries Response Measure:} & \\
  \hline
  \multicolumn{2}{|p{3cm}|}{\bfseries Questions:} &  \\
  \hline
  \multicolumn{2}{|p{3cm}|}{\bfseries Issues:} &  \\
  \hline
\end{tabular}
\caption{Scenario refinement for HS07-...}
\end{center}
\end{table}

% =====================================================================
\section{Evaluation}
% =====================================================================
\section{Architectural Design}
% =====================================================================
\bibliography{paper}
\bibliographystyle{apalike}


\end{document}
