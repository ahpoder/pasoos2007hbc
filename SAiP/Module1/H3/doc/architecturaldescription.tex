%%%
% Architectural Description
%%%
\section{Architectural Description -- Beer Web Store}

The architect for a new Beer Web Store for speciality beers has drawn
the following figure of the architecture together with his customers.
\begin{figure}[h!]
  \centerline{\epsfig{figure=figures/beerstore,scale=0.30,clip=true}}
\end{figure}
The system is to support web users that may purchase beers using a web
browser. The system runs on top of a database (``DB''), contains
domain model (``Data''), an overview page (``Overview''), and a
``Basket'' subpart.

\subsection{Architectural Description - Question 1}
\label{sec:arch_desc_q1}

\begin{question}
Which structures does a system such as the above exhibit?
Give examples of elements and relations pertaining to each
structure
\end{question}

The information depicted in the figure above shows several
software structures of the Beer Web Store to some extent.

\subsubsection{Uses}

The \emph{uses structure} shows how elements are related with regard
to usage. One perspective of the uses structure is that the
elements in the figure are ``Overview'', ``Basket'' and ``Data''
which are all related and the database which is only related
to ``Data''. This shows that the overview page and the basket
has been decoupled from the database which therefore can be
replaced without affecting them.

\subsubsection{Layered}

Another perspective is that the users, the web server and the
database are elements where the relations show that the users
use the web server and the web server uses the database. This
also shows the that the architecture is a
\emph{strictly layered structure} where the users cannot
access the database directly.

\subsubsection{Client-Server}

As there are several users, the elements of the
\emph{strictly layered structure} can also be seen as a
\emph{client-server structure}, where the users are clients
and the web server is a server. The web server can also be
seen as a client to the database server. This would make it
possible to achieve better performance if several web servers
are used to make load balancing of the user requests.

\subsubsection{Deployment}

With regard to a \emph{deployment structure}, the figure shows
that the overview page, the basket and the domain model (elements)
are allocated to (relations) on the web server and the database is
allocated to its own server.

\subsection{Architectural Description - Question 2}

\begin{question}
Which viewpoints are relevant when describing the example?
Give examples of partials views for each viewpoint
\end{question}

Selecting the relevant views shall actually be performed on
the basis of the architectural drivers which have not been
determined for the Beer Web Store. There are however several
models that provide a number of views that are often useful,
e.g. the ``4+1'' model \cite{kruchten1995} and the ``3+1''
model \cite{christensen2004archdesc} and many others.

As described in section \ref{sec:arch_desc_q1} the figure shows
parts of a module viewpoint (uses, layered), component \&
connector viewpoint (client-server) and an allocation viewpoint
(deployment) which are the recommended viewpoints in
\cite{christensen2004archdesc}.

These viewpoints are all relevant to the Beer Web Store example.
The Module viewpoint describes the internals of the web server and how
the functionality is decomposed into units. The Component \& connector
viewpoint describes the runtime relations between the units and the
Allocation viewpoint describes the environment of the running
system.

In the following sections we will give examples on views for each
of the three viewpoints.

\subsubsection{Module Viewpoint}

Figure \ref{fig:ad_package} shows a package diagram of the Beer Web
Store and a decomposition of the web package. The Web package
contains the elements used for the web interface, 
the Overview page, the basket and the domain model. The domain model
depends on a Database package that contains the database fa\c ade and the
database fa\c ade uses some database package library, e.g. MySQL.

\begin{figure}[!htb]
\centerline{\epsfig{figure=figures/ad_package,scale=0.8 }}
\caption{Package diagram for the Beer Web Store}
\label{fig:ad_package}
\end{figure}

\subsubsection{Component \& Connector Viewpoint}

There are only two active components in the Beer Web Store, a web server,
e.g. Apache, and a database server. An active objects diagram will
not be included here.

Figure \ref{fig:ad_sequence} shows a sequence diagram where a user searches
for a an item ``tuborg'' and adds one of the search results to his basket.

\begin{figure}[!htb]
\centerline{\epsfig{figure=figures/ad_sequence,scale=0.6 }}
\caption{Sequence diagram showing a search and adding of an item}
\label{fig:ad_sequence}
\end{figure}

The domain model accesses the database through the database fa\c ade
when needed. This can be expressed in another sequence diagram.
Sequence diagrams showing the checkout including payment should
also be made.

\subsubsection{Allocation Viewpoint}

Figure \ref{fig:ad_allocation} shows a deployment view of the Beer
Web Store where the web server and the database is located on two
different servers. Clients can connect to the Beer Web Server
using a http web browser. The web server and database server
are connected through a local area network.

\begin{figure}[!htb]
\centerline{\epsfig{figure=figures/ad_allocation,scale=0.6 }}
\caption{Deployment view of the Beer Web Store}
\label{fig:ad_allocation}
\end{figure}

The application code (Beer Web) and the database fa\c ade are
both deployed on the web server.

\subsection{Architectural Description - Question 3}

\begin{question}
Argue for benefits and liabilities of describing software
architecture via a box-and-line drawing such as the above
\end{question}

The benefits of a box-and-line drawing is that it can be written and "understood" by almost anyone, and in the initial phases of the architectural design they may be fine. They are fast to draw and can be shown to non-technical stakeholders. It is kind of like simple sign-language - everyone understands nodding and shaking ones head or indicating a direction as long as it is for general information.

Unfortunately the general information is where the advantages stop. For simple sign-language, it may quickly come to an argument if one wishes to know if the indication of "this way" means "a short way that way" or "a long way that way", and the same is true for box-and-line drawings. "How many clients does the system support?". "How strong is the binding indicated by the line - do all lines mean the same?" There is merely a line, so what does that line mean?

Working as an developer/tester/engineer means that everything must be quantifiable in some way, and box-and-line drawings do not offer this unambiguous quantifiable. For this a clearly defined modelling language, without ambiguity as to the meaning of the model, is needed. An example of such a language is Unified Modelling Language(UML), which would allow the box-and-line drawing to be represented as one or more diagrams solving all the ambiguity of the box-and-line drawing with only a single requirement; one must know the language, which is indeed one of the reasons while box-and-line drawings are still widely used - there is no single unambiguous modelling language known by all stakeholders, and when that is the case it may be more confusing then beneficial to spend time making sure the diagram is clear and unambiguous, if that clarity is derived from symbolic notation that the readers (stakeholders) do not understand.

\subsection{Architectural Description - Question 4}
\label{sec:ad_q4}
\begin{question}
Discuss what architectural description would be needed if
the system was to be realized as a Service-Oriented Architecture
\end{question}

In a sense there are already aspects of SOA in the architecture, as the validation of the credit card information is most likely performed using a service on the internet - not that web services and SOA is equivalent, but they do complement each other nicely.

As SOA consists of a collection of (preferably stateless) well-defined services what needs to be described is, which existing (external) services are needed and which new services has to be created. This may be done using the MacKenzie Topic maps, which describe the static dependencies between the services.

The static description is naturally not sufficient, and two other views may be presented to show the Dynamic interrelations for the services that are provided; Visibility and Interaction. The visibility view describes the awareness, willingness and reachability of a service and the interaction view describes how the service can be used (behaviour model) and the structure and semantics of the information that is provided (information model).

