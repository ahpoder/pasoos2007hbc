%%%
% Architectural Design
%%%
\section{Architectural Design -- Beer Web Store}
Consider the following main use case for a Beer Web Store for speciality
beers:
\begin{quote}
  The customer browses the beer catalogue for descriptions and
  reviews. The customer selects appropriate beers for purchase and
  these are added to his shopping basket. When the customer checks
  out, payment is validated through the payment service
\end{quote}
The main architectural drivers are considered to be performance and
availability. The software architect and stakeholders of the system
have based on this outlined the following two central Quality
Attribute Scenarios:
\begin{itemize}
  \item A customer makes a payment to the system in normal mode. The
  payment is processed within 10 seconds
  \item The payment service fails during normal operation. The system
  detects this. The administrator is notified and the system continues
  in degraded mode until the payment service is made available
\end{itemize}

\subsection{Architectural Design - Question 1}

\begin{question}
Give examples of performance tactics respectively availability
tactics that may be applied to the scenario. What are the consequences
of applying these tactics?
\end{question}

This topic has already been breached in Section \ref{sec:qa_q3}, so this section may be seen as a continuation of that.

Assuming that the 10 seconds is an increase in the requirements to the performance, and that the payment service is one that we can actually control (if it is a web-service supplied by a third party (e.g. PBS) then we may have no influence on its performance). 

If this is the case then the performance tactics which may be employed here could be \emph{introduce concurrency}, as some of the validations may be done in parallel, e.g. validate the credit card, prepare shipping information, access logistics system. Naturally this will require a roll-back system if it turns out that the payment did not go through. This approach is acceptable as the performance requirement is only valid in the instance where the payment is valid - how long it takes if the user enters invalid information is less important. 

As there is no clarification of the response measure, it it could be assumed that the requirement must be met 100\% of the time, yet naturally this is not possible. It may however be required that multiple instance of the payment service or validating server are running in order to service multiple purchases simultaneous, as required by the \emph{Availability}.

If the payment is considered of higher priority (\emph{Scheduling Policy} tactic), so that other functionality (e.g. browsing) may be handled slower while the payment is performed, then handling the scheduling of these tasks may also help to meet the 10 second requirement.

Naturally implementing these tactics do not come for free; increasing hardware performance and bandwidth cost money, controlling the scheduling adds complexity and thereby reduce readability and modifiability.

The second scenario focuses mainly on Availability, and actually goes against some of the previos scenarios by stating that the payment service is allowed to be "down" for a longer period of time, pending an administrator. This approach has one big benefit; it is simple. No hot- or cold-swapping of payment services, and the tactic \emph{Spare} may be employed.

If the failure is part of an attack, it is easier to prevent by simply degrading the system and waiting for an administrator (may increase \emph{Security}).

The second scenario does include requirements to fault detection, yet these are already described in Section \ref{sec:qa_q3}, and the price of fault detection is also increased cost of hardware and readability/modifiability.

\subsection{Architectural Design - Question 2}

\begin{question}
The software architect decides to use a layered architectural
style for the system. Discuss quality implications of this choice
\end{question}

Layered styles come in two variants; the strictly layered and the layered. The strictly layered means that the components in a layer may only access other components in the same layer or the layer directly below it. The Layered may access components in the same layer or any layer beneath it.

When a system is Strictly Layered it gives rise to high \emph{Portability} (a for of \emph{Modifiability}), as it is possible to abstract all dependencies of a given layer by simply abstracting the layer directly beneath it, and if anything in a layer changes then only the layer directly above it must be updated.

This also enables simpler \emph{Testability}, as the introduction of test doubles (fake objects, stubs, spys and/or mocks) must only be performed on the in a single layer; the one directly below the one under test.

It may however be difficult to adhere to a Strictly Layered approach, as some functionality may be required on all levels, e.g. the Java Collections, and very often vertical "layers" are introduced to support this all-layer access, which reduces \emph{Modifiability} if the change is in the vertical "layer".

\subsection{Architectural Design - Question 3}

\begin{question}
Outline major elements of a possible component and connector
structure of the Beer Web Store system
\end{question}

There are four major structures in Component-and-Connector:

\begin{itemize}
  \item Process or communicating processes.
	\begin{itemize}
	  \item This structure shows the execution path through a system, and is used to give the engineer an understand about the dynamic behaviour of the system. These structures may help engineers increase Performance and Availability.
	\end{itemize}
  \item Concurrency.
	\begin{itemize}
	  \item This structure is used to see the parallelism in the system, or part of the system, and when a system, or part of a system, is build around this structure it promotes performance. 
	\end{itemize}
  \item Shared data or Repository
	\begin{itemize}
	  \item This is a very good structure to use to illustrate how the data is processed and consumed to/from the database (Server to/from Database Fa\c cade). When a system, or part of a system, is build around this structure it promotes performance and data integrity. 
	\end{itemize}
  \item Client-server
	\begin{itemize}
  		\item This is a very good structure to see the protocols used between the Clients and the Server, and the messages they communicate. When a system, or part of a system, is build around this structure it promotes separation of concern (i.e. Modifiability) and allows for Availability tactics like Load balancing.
	\end{itemize}
\end{itemize}

In the Beer Web Store has two clear C\&C structures, the Shared data structure between the Server and the Database Fa\c cade and the Client-server between the Client and Server (see Figure \ref{fig:module_view}).

The Process or Communicating processes is a more fine-grained structure, and an example of this may be seen in Figure \ref{fig:ad_sequence}.

The Concurrency structure is not used in the overall elements, yet may be employed in a given element to achieve more performance.

