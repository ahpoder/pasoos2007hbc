%%%
% Architectural Design
%%%
\section{Architectural Design -- Beer Web Store}
Consider the following main use case for a Beer Web Store for specialty
beers:
\begin{quote}
  The customer browses the beer catalogue for descriptions and
  reviews. The customer selects appropriate beers for purchase and
  these are added to his shopping basket. When the customer checks
  out, payment is validated through the payment service
\end{quote}
The main architectural drivers are considered to be performance and
availability. The software architect and stakeholders of the system
have based on this outlined the following two central Quality
Attribute Scenarios:
\begin{itemize}
  \item A customer makes a payment to the system in normal mode. The
  payment is processed within 10 seconds
  \item The payment service fails during normal operation. The system
  detects this. The administrator is notified and the system continues
  in degraded mode until the payment service is made available
\end{itemize}

\subsection{Architectural Design - Question 1}

\begin{question}
Give examples of performance tactics respectively availability
tactics that may be applied to the scenario. What are the consequences
of applying these tactics?
\end{question}

This topic has already been breached in Section \ref{sec:qa_q3}, so this section may be seen as a continuation of that.

Asuming that the 10 seconds is an increase in the requirements to the performance, and that the payment service is one that we can actually control (if it is a web-service supplied by a third party (e.g. PBS) then we may have no influence on its performance). 

If this is the case then the performance tactics which may be employed here could be \emph{introduce concurrency}, as some of the validations may be done in parallel, e.g. validate the credit card, prepare shipping information, access logistics system. Naturally this will require a roll-back system if it turns out that the payment did not go through. This approach is acceptable as the performance requirement is only valid in the instance where the payment is valid - how long it takes if the user enteres invalid information is less important. 

As there is no clarification of the response measure, it it could be asumed that the requirement must be met 100\% of the time, yet naturally this is not possible. It may however be requried that multiple instance of the payment service or validating server are running in order to servcie multiple purchases simultanious, as required by the \emph{Availability}.

If the payment is considered of higher priority (\emph{Scheduling Policy} tactic), so that other functionality (e.g. browsing) may be handled slower while the payment is performed, then handling the scheduling of these tasks may also help to meet the 10 second requirement.

Naturally implementing these tactics do not come for free; increasing hardware performance and bandwidth cost money, controlling the scheduling adds complexity and thereby reduce readability and modifiability.

The second scenario focuses mainly on Availability, and actually goes against some of the previos scenarios by stating that the payment service is allowed to be "down" for a longer period of time, pending an administrator. This approach has one big benefit; it is simple. No hot- or cold-swapping of payment services, and the tactic \emph{Spare} may be employed.

If the failure is part of an attack, it is easier to prevent by simply degrading the system and waiting for an administrator (may increase \emph{Security}).

The second scenario does include requirements to fault detection, yet these are already described in Section \ref{sec:qa_q3}, and the price of fault detection is also increased cost of hardware and readability/modifiability.

\subsection{Architectural Design - Question 2}

\begin{question}
The software architect decides to use a layered architectural
style for the system. Discuss quality implications of this choice
\end{question}

\fixme{Answer question 4.2}

\subsection{Architectural Design - Question 3}

\begin{question}
Outline major elements of a possible component and connector
structure of the Beer Web Store system
\end{question}

\fixme{Answer question 4.3}

