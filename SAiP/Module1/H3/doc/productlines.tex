%%%
% Product Lines and Frameworks
%%%
\section{Product Lines -- Beer Web Store}
Consider a Beer Web Store for buying specialty beers. The software
architect has designed a deployment structure of the system as shown
below:
\begin{figure}[h!]
  \centerline{\epsfig{figure=figures/bws-deployment,scale=0.80}}
  \label{fig:observation}
\end{figure}
The main architectural drivers for the system are performance and
modifiability and the following is the main use case for the system:
\begin{quote}
  The customer browses the beer catalogue for descriptions and
  reviews. The customer selects appropriate beers for purchase and
  these are added to his shopping basket. When the customer checks
  out, payment is validated through the payment service
\end{quote}

The company behind the Beer Web Store has decided to build a product
line based on the store so that, e.g., sodas or stamps may be sold in
a new shop or other delivery platforms such as mobile phones may be
used.

\subsection{Product Lines - Question 1}

\begin{question}
How will the choice of a product line approach affect the
architecture of the beer web store?
\end{question}

\fixme{Answer question 5.1}

\subsection{Product Lines - Question 2}

\begin{question}
Discuss the benefits and liabilities of using a product line
approach in relation to the case?
\end{question}

Creating an architecture product line only makes sense if it is believed that it may be used in other products, and the number of products in the pipeline and how well their commonalities may be predicted, helps to determine which approach to use. Naturally the situation is not black and white, as there are many possibilities between the fully evolutionary architecture to the fully initially defined, yet for the purpose of debating advantages and disadvantages the simple black and white is simply easier.

Defining the architecture fully up front means a big investment in time and money in order to create the arcitecture and the first product based on it, so if it is the only product made, then it will go way over budget both in time and money. The idea is that creating the next product will be significantly cheaper, because a very large part of the products functionality is covered by the architecture and very little (if any) redesign has to be done. As more and more products are created based on the architecture their individual budget savings will help pay for the initial investment, eventually leading to a profit compared to building the products from scratch each time.

Letting the architecture evolve means that only the basics are defined up front, only causing a small initial investment, before building the first product. When the second product is to be created more of the architecture is defined based on the parts of the first product, which may be used in the second. Naturally the second product is faster to build than the first, due to reuse, yet not as fast as the second product when the entire architectur was defined in advance. As more products come along the amount of reuse grow, and would eventually reach the same as the one where everything was defined in advance.

If the architecture is used to define a Product Line it becomes even more pronounced, and the above two approaches are defined as Heavyweight and Lightweight, respectively. A statistical analysis of these two approaches, compared to simply building the products from scratch, can be shown in figure xx.

\clearpage

\begin{figure}[!htb]
\centerline{\epsfig{figure=figures/product-line,scale=1.0 }}
\caption{Heavy-weight vs. Light-weight}
\label{fig:product-line}
\end{figure}

Deciding which approach to use, is more than just a matter of choosing between Heavyweight and Lightweight (or fully defined or evolutionary) based on the number of expected products, it is also a matter of how well the common architecture of the future products can be predicted, for if the architecture needs to be changed from product to product then all the advantage is lost.

Based on this the Beer Web Store architect must have decided that the products in the pipe-line and the Beer Web Store share a common architecture and that the future products are so well know that the architecture can be fully specified in advance.

\subsection{Product Lines - Question 3}

\begin{question}
Relate Service-Oriented Architecture and Product Lines?
\end{question}

\fixme{Answer question 5.3}

