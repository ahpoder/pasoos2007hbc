%%%
% Product Lines and Frameworks
%%%
\section{Product Lines -- Beer Web Store}
Consider a Beer Web Store for buying specialty beers. The software
architect has designed a deployment structure of the system as shown
below:
\begin{figure}[h!]
  \centerline{\epsfig{figure=figures/bws-deployment,scale=0.80}}
  \label{fig:observation}
\end{figure}
The main architectural drivers for the system are performance and
modifiability and the following is the main use case for the system:
\begin{quote}
  The customer browses the beer catalogue for descriptions and
  reviews. The customer selects appropriate beers for purchase and
  these are added to his shopping basket. When the customer checks
  out, payment is validated through the payment service
\end{quote}

The company behind the Beer Web Store has decided to build a product
line based on the store so that, e.g., sodas or stamps may be sold in
a new shop or other delivery platforms such as mobile phones may be
used.

\subsection{Product Lines - Question 1}

\begin{question}
How will the choice of a product line approach affect the
architecture of the beer web store?
\end{question}

When considering a product line approach for our Beer Web Store, we first need to identify the scope of our product line. In this case the product line is defined to have a narrow scope. The product line supports a web enabled store that sells quantifiable products (i.e. not services). The products may however have different details associated, e.g. beers have a alcohol volume, stamps do not. The web stores must support different user interfaces (but all web based) and on line payment methods.

All these variations and commonalities needs to be identified. The variations needs to be separated so that they either can be created specifically for the actual web store, or so that a common component can be configured to handle the specific product.

The following variation points can be identified from the description above:

\begin{description}
	\item[Products:] The products of the web stores are the main variation point when creating this product line. Some product details will be relevant to all stores, e.g. a product price, and others will be specific to the type of product sold, e.g. alcohol volume as stated above.
	\item[User interface:] It must be possible to vary the user interface layout with regard to colors and structure of the web page in order to sell the specific product the best possible way. Furthermore the user interface must be decoupled from the handling of products in order to support several types of web interfaces such as regular browsers and mobile devices.
	\item[Payment form:] On line payment methods must be supported, but all stores may not want to support all types of credit cards. The supported payment forms must therefore be a variation point in the product line.
\end{description}

\begin{figure}[!htb]
\centerline{\epsfig{figure=figures/pl_package_overview,scale=0.6 }}
\caption{Package diagram for the web store product line}
\label{fig:pl_package_overview}
\end{figure}

Figure \ref{fig:pl_package_overview} shows an example of a package diagram for a web
store product line. It shows that the user interface and the product information has
been separated from the product handling and put in separate packages. The user
interface may be entirely specific to each web store or the package may contain
predefined user interfaces that can be configured e.g. using style sheets. A reactive
approach can be used to develop more web interfaces when the need for more features
arises. The user interface depends on the product handling package that can handle
all kinds of products. The products package can provide an interface that the user
interfaces and product handling module can depend on. There need to be some way to
define product specific information that can be retrieved from and searched for in
the database and presented on the user interface. This concept is not defined any
further in this report.

The dependencies to from the user interface and product handling to the products package can e.g. be solved at compile time.

\begin{figure}[!htb]
\centerline{\epsfig{figure=figures/pl_package_ph,scale=0.6 }}
\caption{Decomposition of the Product Handling Package for the web store product line}
\label{fig:pl_package_ph}
\end{figure}

In figure \ref{fig:pl_package_ph} the Product Handling package has been decomposed
into smaller components. The ``Basket'', ``Overview'' and ``Domain Model'' from the
original Beer Web Store has been generalized so that they do not contain any product
specific handling. A ``Payment'' package has been created that shall support all
needed payment methods. When need for a new payment method arises, this must be added
to the package. The supported payment methods for a specific web store must be
configured, e.g. at compile time.


\subsection{Product Lines - Question 2}
\label{sec:pl_q2}
\begin{question}
Discuss the benefits and liabilities of using a product line
approach in relation to the case?
\end{question}

Creating an architecture product line only makes sense if it is believed that it may be used in other products, and the number of products in the pipeline and how well their commonalities may be predicted, helps to determine which approach to use. Naturally the situation is not black and white, as there are many possibilities between the fully evolutionary architecture to the fully initially defined, yet for the purpose of debating advantages and disadvantages the simple black and white is simply easier.

Defining the product line architecture fully up front means a big investment in time and money in order to create the architecture and the first product based on it, so if it is the only product made, then it will go way over budget both in time and money. The idea is that creating the next product will be significantly cheaper, because a very large part of the products functionality is covered by the product line and very little (if any) redesign has to be done. As more and more products are created based on the product line their individual budget savings will help pay for the initial investment, eventually leading to a profit compared to building the products from scratch each time.

Letting the product line evolve means that only the basics are defined up front, only causing a small initial investment, before building the first product. When the second product is to be created more of the product line is defined based on the parts of the first product, which may be used in the second. Naturally the second product is faster to build than the first, due to reuse, yet not as fast as the second product when the entire product line was defined in advance. As more products come along the amount of reuse grow, and would eventually reach the same as the one where everything was defined in advance.

The above two approaches are defined as Heavyweight and Lightweight, respectively. A statistical analysis of these two approaches, compared to simply building the products from scratch, is shown in Figure \ref{fig:product-line}.

\begin{figure}[!htb]
\centerline{\epsfig{figure=figures/product-line,scale=0.9 }}
\caption{Heavy-weight vs. Light-weight \cite{mcgregor2002}}
\label{fig:product-line}
\end{figure}

Deciding which approach to use, is more than just a matter of choosing between Heavyweight and Lightweight (or fully defined or evolutionary) based on the number of expected products, it is also a matter of how well the common architecture of the future products can be predicted, for if the architecture needs to be changed from product to product then all the advantage is lost.

Based on this the Beer Web Store architect must have decided that the products in the pipe-line and the Beer Web Store share a common architecture and that the future products are so well known that the architecture can be fully specified in advance.

\subsection{Product Lines - Question 3}

\begin{question}
Relate Service-Oriented Architecture and Product Lines?
\end{question}

As mentioned in Section \ref{sec:qa_q4} part of SOA is about generalization, just like product lines are. Taking the example given above, having the architecture support mobile phones (i.e. WAB) means that the server must be able to supply a scaled down result of a search as well as the fully blown. The server must thereby handle different types of clients.

The requirement that other product's such as other stamps or soda-pop should be supported, means that the server's domain model and database must support general elements as opposed to just beer elements.

This generalization fits well with SOA, as it is not that clever to make a ``Beer purchasing Service'', but far better to create a ``Purchasing service'', which operate on elements. This will allow any company, who has products they wish to sell, to use this service. This is the same generalization requirement that apply to the Product line architecture, where the architecture must handle the purchase of an arbitrary product, be it stamps, soda-pop or beer. It is possible that the architecture requires that it is a quantifiable product, yet this is not an unlikely requirement for a service either.

It may now seem like SOA is the answer to creating a product line, yet this is not the case. It is very possible to create very specific services, like the ``Beer look-up service'', and this may be just fine for a Beer Web Store based on SOA, yet this is not a product line. The same architectural consideration must be done when creating the elements of a product line and define their interrelations, as when defining the services and their interrelations. The only ``structure'' that SOA imposes by definition is the loose coupling between the user and the service.
