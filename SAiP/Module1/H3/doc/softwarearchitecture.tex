%%%
% Architectural Description
%%%
\section{Software Architecture -- Beer Web Store}

A new Beer Web Store for specialty beers has the following main use
case:
\begin{quote}
  The customer browses the beer catalogue for descriptions and
  reviews. The customer selects appropriate beers for purchase and
  these are added to his shopping basket. When the customer checks
  out, payment is validated and he is presented with a receipt
\end{quote}
As a first step, the software architect of this systems has decided to
use a layered architectural style (with clients, server, and a
database in separate layers). He has decided that the communication to
the database will be done via a Fa\c cade pattern. The system will be
built in Java.

\subsection{Software Architecture - Question 1}
\begin{question}
Outline how you would approach the tasks of creating the
architecture for the Beer Web Store. Consider, e.g., which steps
would you takes and in which order?
\end{question}

Naturally there are many ways to create an architecture, yet as the focus is on the architecture then ADD (Architecture Driven Design) is an obvious choice. ADD is merely a method that employs many of the techniques related to Software Architecture in a predefined structure.

\subsubsection{Choose the module to decompose}
As this is the beginning of the architectual design then the module is the entire system. In later iterations the sub-components (the layers, the components in the layers, etc.) will be decomposed.

\subsubsection{Refine the module according to the following steps}

\begin{description}
    \item[a. Choose the architectual drivers]
  Here the quality attributes of the system would be defined by the stakeholders of the system and then a vote would be used to determine which of the quality attributes are architectual drivers. The quality attributes would be expressed as quality attribute scenarios.
    \item[b. Choose an architectual pattern]
  Based on the architectual drivers selected in 1, use the mapping in Bass et al. as well as other literature and experience to determine some tactics for achieving the architectual drivers for the system. Then select some architectual patterns to realize the tactics.
    \item[c. Instantiate modules using multiple views]
  Based on the tactics and patterns found above the architecture for the system may be expressed using UML. It is \emph{always} required to use multiple views in order to properly describe the architecture. A good approach is the "3 + 1" structure, which uses three viewpoints; Module viewpoint, Component \& Connector viewpoint and Allocation viewpoint. These three viewpoint is expressed in UML using Package diagrams, Object and sequence diagrams, and deployment diagrams respectively. These diagrams should in the first iteration only look at relations and interactions between the main components (the layers).
    \item[d. Define interfaces]
  Define the interfaces between the main components (the layers).
    \item[e. Verify scenarios and use cases]
  Make sure that the steps a - d covered all the parts of the scenarios. If not include the missing parts and updates the scenarios/architecture is inconsistencies are found.
\end{description}
\subsubsection{Repeat the steps above for every module that needs further decomposition}
The first iteration only look at the overall architecture (the leation and communication between layers). The following iterations will decompose the individual layers and as needed the individual components in the layers.

\subsection{Software Architecture - Question 2}

\begin{question}
Give concrete examples of what elements, relations, and structures as defined in \cite{bass2003sa} could be in relation to an architecture for the Beer Web Store.
\end{question}

\subsubsection{Elements}
Based on the present decomposition the elements that are know are the layers, where each layer is an element as described below.
\begin{itemize}
    \item[Clients] The clients access the server in order to perform some action.
    \item[Server] The server is accessed by the clients and when needed access the database via its Fa\c cade pattern.
    \item[Database] The database exposes its interface to the server via a Fa\c cade pattern.
\end{itemize}
The above is quickly sketched in the box-and-line drawing below, where the layering may also be seen. 
\begin{figure}[!htb]
\centerline{\epsfig{figure=figures/module_view,scale=1.0 }}
\caption{Element overview}
\label{fig:module_view}
\end{figure}
\clearpage

\subsubsection{Relations}
The figure xx shows some of the relations, namely the statical, yet it is also important to know the dynamical behaviour. These may be expressed by e.g. a sequence diagram, as shown in Figure yy. In these it may be seen that the Client and Server has a week semantic relation through the http protocol, and the Server and Database has a strong syntactic relation from a compiled compatibility. 

\begin{figure}[!htb]
\centerline{\epsfig{figure=figures/sequence,scale=1.0 }}
\caption{Relation overview}
\label{fig:sequence}
\end{figure}

\subsubsection{Structures}

There are many structures which may be used to model a software system. Common for all of them is that they present a certain view of the system. Very rarely is a single structure sufficient to sufficiently express the architecture of a system. For this reason many structures are combined, each structure being represented by one or more views, to create a sufficient representation of the software system.

The most popular combination of structures are the Module, the Component-and-Connector and the Allocation. Each of these three structures focus on a specific aspect of the software architecture; the static elements and their relation, the elements dynamic relations and the elements deployment on hardware, respectively. 

In each structure there is a number of views. Which view, and indeed which structure, that apply is quite dependent on not only the system, but also the iteration of the architectual analysis. In the first iteration of the beer-store it would make sense to choose more overall views, e.g. the Module Layered or Module Decomposition, whereas e.g. the Module Class would be too fine-grained for the first iteration. For the C\&C it would make sense to start with the Client-Server, and finally for the Allocation the Deployment view may give a good indication of the physical location and requirements of the different elements.

\subsection{Software Architecture - Question 3}

\begin{question}
Apply \cite{perrywolf1992}'s model of software architecture as
\begin{quote}
	{\it Software Architecture = \{Elements, Form, Rationale\}}
\end{quote}
to the Beer Web System
\end{question}

\fixme{Answer question 1.3}

\subsection{Software Architecture - Question 4}

\begin{question}
Reflect on what happens if the words ``software'' and
``computing'' are removed from the definition of 'software
architecture' in [Bass et al., 2003]
\end{question}

\fixme{Answer question 1.4}

\subsection{Software Architecture - Question 5}

\begin{question}
The architect decides to create a full architecture
description before embarking on any implementation of the
system. Discuss pros and cons of taking that approach
\end{question}

\fixme{Answer question 1.5}
