\documentclass[a4paper,10pt]{article}

\usepackage[latin1]{inputenc}
\usepackage{epsfig}
\usepackage{graphicx}
\usepackage{url}
\usepackage{times}
\usepackage{rotating}

\begin{document}

\title{Exercise 1: Software Architecture Description of the HS07 System}

\author{
  Anders H Poder, Jesper Dalberg, Lars Kringelbach\\\\
  Department of Computer Science, University of Aarhus\\
  Aabogade 34, 8200 {\AA}rhus N, Denmark\\\\
  \makeatletter
  \texttt{Group 11 - Kilo}\\
  \texttt{19951439, 20074976, 20074842}\\
  \texttt{\{ahp, jdalberg, u074842\}@daimi.au.dk}
}

\date{2008-02-04 13:32}

\maketitle

% =====================================================================
\begin{abstract}
  The HS07 system implements a closed-loop control of the heating in a
  private home. It monitors thermometers in the home, and based on
  measurements HS07 adjusts radiators in the home. This report gives a
  software architecture description of an architectural prototype of
  the HS07 system. The techniques used for architectural description
  are taken from \cite{christensen2004archdesc}.
\end{abstract}

% =====================================================================
\section{Introduction}

Figure~\ref{fig:hs07} shows a schematic overview of HS07 in a
home. The home may be accessed by the home owner from the outside
through the HS07 gateway. The HS07 gateway also monitors and controls
the home.
\begin{figure}[!htb]
\centerline{\epsfig{figure=figures/hs07,scale=0.4 }}
\caption{HS07 in a home}
\label{fig:hs07}
\end{figure}

HS07 includes sensor and actuator hardware which runs on an embedded Java virtual
machine with standard software.

% =====================================================================
\section{Architectural Requirements}

For our purposes there is one main use case for the HS07 system:
\begin{quote}
  \emph{Control Temperature}: The gateway collects measurements from
  thermometers and reports this to radiators that then control the
  temperature.
\end{quote}

The major driving qualities attributes of the HS07 system
are\footnote{These qualities will be operationalized in Exercise 2}:

\begin{itemize}
\item \emph{Performance.} HS07 should be performant so that a large
  number of thermometers and radiators may be part of the system.
\item \emph{Modifiability.} It must be possible to modify HS07 to
  include new types of sensors and actuators.
%  \item $<<$Extra quality requirement that you consider important $>>$
\end{itemize}


% =====================================================================
\section{Architectural Description}

% ---------------------------------------------------------------------
\subsection{Module Viewpoint}

The module viewpoint of the HS07 system.

\begin{figure}[!htb]
\center {\includegraphics[viewport=0 10 600 500,scale=0.5]{figures/mv.pdf}}
\caption{HS07 Package Diagram}
\label{fig:mv}
\end{figure}

Figure~\ref{fig:mv} shows the packages of the HS07 system, and how they interact 
with java packages used. The Sensor, Gateway and Actuator packages do not have
any knowledge of each other. They are all services that are able to respond to
HTTP queries. They uses the Invoker to interact with the other modules' HTTP
server. The interactions between the components are described in section
\ref{sec:component}. The Java packages Net, Jetty and Http are used for the
HTTP servers and HTTP requests.

% ---------------------------------------------------------------------
\subsection{Component \& Connector Viewpoint}
\label{sec:component}

The Component \& Connecter view consists of an Active Objects diagram and a
sequence diagram.

The Active Objects diagram in figure \ref{fig:cc_ao} shows the active objects
of the HS07 system, and how they interact. The Gateway accesses the thermometers
by requesting the temperature repeatedly from all registered TemperatureServices.
The measured temperature is then published to all registered observers. The
RadiatorService and ThermometerService registers to the Gateway as Observers and
Servers respectively through the GatewayService. It is also possible to register
to the Gateway from an internet connection.

\begin{figure}[!htb]
\center {\includegraphics[viewport=0 10 600 420,scale=0.5]{figures/cc_ao.pdf}}
\caption{HS07 Active Objects}
\label{fig:cc_ao}
\end{figure}
The sequence diagram illustrates the protocol of the interactions involved in the setup and the execution of the Gateway. It consists of a registration sequence and an infinite loop of retrieving the temperature of all registered themometers and nitifying all observers (in this case the Radiators). The example is slightly simplified to improve readability, as the registerThermometer, registerObserver, getTemperature and notify is actually network calls performed using the Invoker and the respective servers.
\begin{figure}[!htb]
\includegraphics[angle=90,scale=1.2]{figures/sequence.pdf}
\caption{HS07 Sequence Diagram}
\label{fig:sequence}
\end{figure}
\clearpage


% ---------------------------------------------------------------------
\subsection{Allocation Viewpoint}

The allocation viewpoint illustrates how components are deployed in actual processes
within the HS07 system.

\begin{figure}[!htb]
\center {\includegraphics[viewport=0 10 700 350,scale=0.4]{figures/deployment.pdf}}
\caption{HS07 Allocation Diagram}
\label{fig:allocation}
\end{figure}


% =====================================================================
\section{Discussion}
\subsection{Strength and Limitations of the approach}
%$<<$What are the strengths and limitations of this approach to architectural description judging from this case?$>>$
\subsection{Strengths}
\begin{itemize}
\item UML is a language that is unambigious and known by many.
\item The three viewpoints are tried and proven, and due to this fact many developers are used to thinking in this way and can generate an understanding of the system based on these views.
 \item Having a well-understandable architecture to go by, makes it easier for multiple developers to work in parallel with a common understanding of the priorities inherint in the architecture.
\end{itemize}

\subsection{Limitations}
\begin{itemize}
\item Just like with any other language it only makes sense to those who know the language. 
\item Choosing three views to focus on, due to the fact that they fit most of the time, means that the focus will sometimes be wrong.
\end{itemize}

\subsection{Not covered aspects}
%$<<$Are there aspects of the software architecture that have not been properly described?$>>$
\begin{itemize}
\item \emph{Stakeholders and their concerns}. This encompasses both users/aquirers and developers. \item \emph{Ffeasibility}
\item \emph{Maintainablity}
\item \emph{Security} How is the Gateway protected? What protection is there against attack and inproper use? 
\item \emph {Safety} How is synchronization done? What insurances is there that the system cannot mailfunction and overheat a radiator or turn on a radiator which has been turned off deliberately because flammable material is placed near it?
\item \emph{Availability} What is the up-time of the HTTP server that the system run on, both embedded and PC?
\item \emph{Testability} What is the tesability of the components? How easily is it to replace the actual HTTP servers with test doubles?
\end{itemize}

\subsection{IEEE Conformance}

%$<<$For the architectural description above, discuss what (if anything) should be changed or added for it to comply with the IEEE recommended practice for architectural description$>>$

The following are the problems identified in order to comply with \cite{ieeerecommendedpractice}.

Clause 5.2: The stakeholders and their precise concerns need to be identified for each view. 

Clause 5.5: Consistency checks across all existing views.

\subsection{Perry \& Wolf Considerations}
%$<<$Consider the definition of software architecture by [Perry and Wolf, 1992]. Discuss what the 'elements', 'form', and 'rationale' according to this definition would be for the HS07 system$>>$

\cite{perrywolf1992} defines "elements" as three different classes, data elements, processing elements and connecting elements. In HS07 the data elements are temperatures, the processing element is the gateway and the connecting element is HTTP. The "form", as described by Perry and Wolf, is the choice of the observer pattern in HS07, and the "rationale" for doing so are defined by the Achitectural Requirements described elsewhere in this document.

% =====================================================================
\bibliography{paper}
\bibliographystyle{apalike}


\end{document}
