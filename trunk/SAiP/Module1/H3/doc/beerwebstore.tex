%%%%
%%%% Beer Web Store
%%%%
\section{Software Architecture in Practice Case: Beer Web Store}

%%%
% Architectural Description
%%%
\newpage
\subsection{Software Architecture -- Beer Web Store}
A new Beer Web Store for specialty beers has the following main use
case:
\begin{quote}
  The customer browses the beer catalogue for descriptions and
  reviews. The customer selects appropriate beers for purchase and
  these are added to his shopping basket. When the customer checks
  out, payment is validated and he is presented with a receipt
\end{quote}
As a first step, the software architect of this systems has decided to
use a layered architectural style (with clients, server, and a
database in separate layers). He has decided that the communication to
the database will be done via a Fa\c cade pattern. The system will be
built in Java.

Discuss:
\begin{itemize}
    \item Outline how you would approach the tasks of creating the
    architecture for the Beer Web Store. Consider, e.g., which steps
    would you takes and in which order?
    \item Relate the definition of 'software architecture' in [Bass et
    al., 2003] to the elements of your design as outlined above. Give
    concrete examples of elements, relations, and structures.
    \item Apply [Perry and Wolf, 1992]'s model of software architecture as
      \begin{quote}
	{\it Software Architecture = \{Elements, Form, Rationale\}}
      \end{quote}
      to the Beer Web System
    \item Reflect on what happens if the words ``software'' and
    ``computing'' are removed from the definition of 'software
    architecture' in [Bass et al., 2003]
    \item The architect decides to create a full architecture
    description before embarking on any implementation of the
    system. Discuss pros and cons of taking that approach
\end{itemize}

%%%
% Architectural Description
%%%
\newpage
\subsection{Architectural Description -- Beer Web Store}
The architect for a new Beer Web Store for specialty beers has drawn
the following figure of the architecture together with his customers.
\begin{figure}[h!]
  \centerline{\epsfig{figure=figures/beerstore,scale=0.30,clip=true}}
\end{figure}
The system is to support web users that may purchase beers using a web
browser. The system runs on top of a database (``DB''), contains
domain model (``Data''), an overview page (``Overview''), and a
``Basket'' subpart.

Discuss:
\begin{itemize}
    \item Which structures does a system such as the above exhibit?
    Give examples of elements and relations pertaining to each
    structure
    \item Which viewpoints are relevant when describing the example?
    Give examples of partials views for each viewpoint
    \item Argue for benefits and liabilities of describing software
    architecture via a box-and-line drawing such as the above
    \item Discuss what architectural description would be needed if
    the system was to be realized as a Service-Oriented Architecture
\end{itemize}


%%%
% Quality Attributes
%%%
\newpage
\subsection{Quality Attributes -- Beer Web Store}

Consider the following main use case for a Beer Web Store for specialty
beers:

\begin{quote}
  The customer browses the beer catalogue for descriptions and
  reviews. The customer selects appropriate beers for purchase and
  these are added to his shopping basket. When the customer checks
  out, payment is validated through the payment service
\end{quote}
The main architectural drivers are considered to be performance and
availability. Moreover, the software architect for the system has
created the following box-and-line drawing documenting his initial
ideas of the system:
\begin{figure}[h!]
  \centerline{\epsfig{figure=figures/beerstore,scale=0.30,clip=true}}
\end{figure}

Discuss:
\begin{itemize}
    \item Describe technique(s) for architectural requirements capture
    that are applicable to the above case
    \item Give feasible architectural requirements for availability
    and performance for the Beer Web Store using such techniques (at
    least one for each quality attribute)
    \item Argue how tactics and/or styles may be used to resolve the requirements
    \item Discuss where Service-Oriented Architecture can play a role
    with respect to the Beer Web Store? Consider the quality implications
    \item Reflect upon what the role of quality attributes in software
    architecture is
\end{itemize}

%%%
% Architectural Design
%%%
\newpage
\subsection{Architectural Design -- Beer Web Store}
Consider the following main use case for a Beer Web Store for specialty
beers:
\begin{quote}
  The customer browses the beer catalogue for descriptions and
  reviews. The customer selects appropriate beers for purchase and
  these are added to his shopping basket. When the customer checks
  out, payment is validated through the payment service
\end{quote}
The main architectural drivers are considered to be performance and
availability. The software architect and stakeholders of the system
have based on this outlined the following two central Quality
Attribute Scenarios:
\begin{itemize}
  \item A customer makes a payment to the system in normal mode. The
  payment is processed within 10 seconds
  \item The payment service fails during normal operation. The system
  detects this. The administrator is notified and the system continues
  in degraded mode until the payment service is made available
\end{itemize}

Discuss:
\begin{itemize}
\item Give examples of performance tactics respectively availability
tactics that may be applied to the scenario. What are the consequences
of applying these tactics?
\item The software architect decides to use a layered architectural
style for the system. Discuss quality implications of this choice
\item Outline major elements of a possible component and connector
structure of the Beer Web Store system
\end{itemize}


%%%
% Product Lines and Frameworks
%%%
\newpage
\subsection{Product Lines -- Beer Web Store}
Consider a Beer Web Store for buying specialty beers. The software
architect has designed a deployment structure of the system as shown
below:
\begin{figure}[h!]
  \centerline{\epsfig{figure=figures/bws-deployment,scale=0.80}}
  \label{fig:observation}
\end{figure}
The main architectural drivers for the system are performance and
modifiability and the following is the main use case for the system:
\begin{quote}
  The customer browses the beer catalogue for descriptions and
  reviews. The customer selects appropriate beers for purchase and
  these are added to his shopping basket. When the customer checks
  out, payment is validated through the payment service
\end{quote}

The company behind the Beer Web Store has decided to build a product
line based on the store so that, e.g., sodas or stamps may be sold in
a new shop or other delivery platforms such as mobile phones may be
used.

Discuss:
\begin{itemize}
\item How will the choice of a product line approach affect the
  architecture of the beer web store?
  \item Discuss the benefits and liabilities of using a product line
    approach in relation to the case?
  \item Relate Service-Oriented Architecture and Product Lines?
\end{itemize}

%%% Local Variables: 
%%% mode: latex
%%% TeX-master: "document"
%%% End: 
